%%%%%%%%%%%%%%%%%%%%%%%%%%%%%%%%%%%%%%%%%
% Medium Length Professional CV
% LaTeX Template
% Version 2.0 (8/5/13)
%
% This template has been downloaded from:
% http://www.LaTeXTemplates.com
%
% Original author:
% Trey Hunner (http://www.treyhunner.com/)
%
% Important note:
% This template requires the resume.cls file to be in the same directory as the
% .tex file. The resume.cls file provides the resume style used for structuring the
% document.
%
% 
% 
%%%%%%%%%%%%%%%%%%%%%%%%%%%%%%%%%%%%%%%%%

%----------------------------------------------------------------------------------------
%	PACKAGES AND OTHER DOCUMENT CONFIGURATIONS
%----------------------------------------------------------------------------------------

\documentclass{resume} % Use the custom resume.cls style

\usepackage[ top=1in, bottom=1in, left=.75in,right=.75in]{geometry} % Document margins
\usepackage{tipa}
\usepackage[hidelinks]{hyperref}
\usepackage{url}
\usepackage{fontawesome}
\usepackage{fancyhdr}
\usepackage{lastpage}
\usepackage{eurosym}
\pagestyle{fancy}


\lhead{Joshua Meyer}
\rhead{Curriculum Vitae}
%% \cfoot{Page \thepage\ of \pageref{LastPage}}
\cfoot{Page \thepage}
\renewcommand{\footrulewidth}{0.4pt}
\setlength{\headheight}{15pt} 
\thispagestyle{empty}

\name{Josh Meyer} % Your name

\begin{document}

\vspace{-.5cm}
\begin{center}
  \href{mailto:joshua.richard.meyer@gmail.com}{\nolinkurl{joshua.richard.meyer@gmail.com}}\\
  \href{https://jrmeyer.github.io}{\textbf{jrmeyer.github.io}} \\
  \href{https://github.com/JRMeyer}{\faGithub} \hspace{.25cm} \href{https://www.linkedin.com/in/josh-r-meyer/}{\faLinkedin} \\

\vspace{.05cm}

\textit{I'm a PhD candidate passionate about Machine Learning and open-source language technologies. \\ My educational background is in Computational Linguistics, Theoretical Linguistics, Natural Language Processing, Statistics, and Cognitive Science. My PhD thesis focuses on Speech Technology, and Automatic Speech Recognition in particular. I work to develop approaches for training more robust Deep Neural Networks for Speech Recognition, using Multi-Task Learning and Transfer Learning. \\ My research is linguistically informed, but based in machine learning methods. }

\end{center}


%----------------------------------------------------------------------------------------
%	EDUCATION SECTION
%----------------------------------------------------------------------------------------
%\begin{minipage}{\textwidth}
\begin{rSection}{Education}
{\bf University of Arizona} \hfill {Expected 2019} \\ 
Ph.D. in Linguistics (focus: \textit{computational linguistics}) \hfill {\em Tucson, AZ}

{\bf University of Arizona} \hfill {May 2015} \\ 
M.A. in Linguistics \hfill {\em Tucson, AZ}

{\bf Seton Hall University} \hfill {May 2012} \\ 
B.A. in Liberal Studies  \hfill {\em South Orange, NJ} \\
\end{rSection}

%\end{minipage}
%----------------------------------------------------------------------------------------

\begin{rSection}{Industry Experience}
  
{\bf Mozilla Machine Learning} \hfill {October 2018 --- Present} \\ 
NSF-Sponsored Internship \hfill {} \\
\textit{Developing end-to-end Automatic Speech Recognition techniques} \hfill {} \\
\textit{with Transfer Learning and deep neural networks (i.e. DeepSpeech).} \hfill {} \\
  
{\bf Various Companies} \hfill {2016 --- 2018} \\ 
Kaldi \& ASR Consultant  \hfill {} \\
\textit{Optimized Automatic Speech Recognition pipelines}   \hfill {} \\
\textit{for various companies using the Kaldi toolkit.}   \hfill {} \\
\end{rSection}


%----------------------------------------------------------------------------------------

\begin{rSection}{Open Source Work}

  {\textbf{Mozilla DeepSpeech}} {\hfill \textit{Added Transfer Learning branch} {\hspace{2.5cm} \href{https://github.com/mozilla/DeepSpeech/tree/transfer-learning}{\textbf{[Code]}}}} \\
\vspace{-.35cm}

{\textbf{Multi-Task Kaldi}} {\hfill \textit{Multi-Task Acoustic Modeling with Neural Networks} {\hspace{2.5cm} \href{https://github.com/JRMeyer/multi-task-kaldi}{\textbf{[Code]}}}} \\
\vspace{-.35cm}

{\textbf{eSpeak NG}} {\hfill \textit{Added Kyrgyz language to speech synthesizer} {\hspace{2.5cm} \href{https://github.com/rhdunn/espeak/commits?author=JRMeyer}{\textbf{[Code]}}}}\\
\vspace{-.35cm}

\end{rSection}

\begin{minipage}{\textwidth}

\begin{rSection}{Technical Strengths}
\vspace{.25cm}

\begin{tabular}{ @{} >{\bfseries}l @{\hspace{6ex}} l }
Computer Languages & Python, Bash, \textsc{MATLAB}, Perl, R, C++ (\textit{some knowledge}) \\
Other & TensorFlow, Kaldi, virtualenv, scikit-learn, pandas,\\
      & nltk, matplotlib, kenlm, Linux, AWS, git, GitHub \\
\end{tabular}
\end{rSection}

\end{minipage}
\end{document}

